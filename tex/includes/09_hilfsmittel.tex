\section{Hilfsmittel}
\label{sec:hilfsmittel}

% Allgemeine Hilfsmittel
Die genutzten Hilfsmittel sind in Abbildung \ref{tab:tools} aufgeführt.

Sonstige benutzte Python-Module sind in der Datei \texttt{python/requirements.txt} zu finden.

% IDE, Compiler, Debugger, etc. for python development
\begin{figure}[H]
    \centering
    \begin{tabular}{|l|l|l|}
        \hline
        \textbf{Typ}            & \textbf{Tool}               & \textbf{Hinweise}                          \\
        \hline
        \hline
        \textbf{CPU}            & \texttt{AMD Ryzen 5 3600}   & \texttt{6x 3.60 GHz}                       \\
        \hline
        \textbf{RAM}            & \texttt{32 GB}              & \texttt{DDR4-3200}                         \\
        \hline
        \hline
        \textbf{Betriebssystem} & \texttt{Ubuntu}             & \texttt{22.04.1 LTS}                       \\
        \hline
        \textbf{Kernel}         & \texttt{Linux}              & \texttt{5.15.74.2-microsoft-standard-WSL2} \\
        \hline
        \textbf{IDE}            & \texttt{Visual Studio Code} & \texttt{1.73.1}                            \\
        \hline
        \textbf{Compiler}       & \texttt{Python}             & \texttt{3.10.8}                            \\
        \hline
        \textbf{Linter}         & \texttt{pylint}             & \texttt{2.15.7}                            \\
        \hline
        \textbf{Dokumentation}  & \texttt{pdoc}               & \texttt{12.3.0}                            \\
        \hline
        \textbf{UML-Diagramme}  & \texttt{PlantUML}           & \texttt{V1.2022.13}                        \\
        \hline
    \end{tabular}
    \caption{Hilfsmittel}
    \label{tab:tools}
\end{figure}
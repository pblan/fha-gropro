\section{Abweichungen vom ursprünglichen Konzept}
\label{sec:abweichungen}

% Abweichungen zum urprünglichen Konzept
Im Laufe der Entwicklung des Programms sind einige Abweichungen vom ursprünglichen Konzept aufgetreten.
Im Allgemeinen sind diese Abweichungen aufgrund von erkannter Redundanz oder aufgrund von technischen Limitierungen bzw. Verbesserungen entstanden.

Redundant waren beispielsweise:
\begin{itemize}
    \item Abspeichern von Start- und Endkoordinaten in der \texttt{Movement}-Klasse, da diese Informationen bereits in den zugehörigen \texttt{Phasen} enthalten sind.
    \item Analog das Abspeichern von Distanz und Dauer in der \texttt{Movement}-Klasse.
    \item Dimension, Frequenz und Queue sind nun sinnvoller gespeichert als im usprünglichen Entwurf.
\end{itemize}

Von den Algorithmen wurde nur minimal abgewichen.
Teilweise wurden leicht andere Formeln zur Berechnung der Bewegungen verwendet um potentielle Rundungsfehler durch Operationen auf Gleitkommazahlen zu minimieren.

Insbesondere bei der Berechnung von Distanzen, Zeiten und Geschwindigkeiten wurde bevorzugt auf die Verwendung von \texttt{numpy}-Arrays und -funktionen zurückgegriffen.
Diese sind deutlich schneller, einfacher und robuster als die selbstgeschriebenen Alternativen.
\section{Benutzeranleitung}
\label{sec:benutzeranleitung}

\subsection{Installation}
\label{ssec:installation}

\subsubsection{Installation mit conda}
\label{sssec:installation_mit_conda}

Zur Installation mit \texttt{conda} wird eine verwendbare \texttt{conda}-Instanz benötigt.

Empfohlen wird Miniconda: \url{https://docs.conda.io/en/latest/miniconda.html}.

Die Installation funktioniert dann wie folgt:
\begin{enumerate}
    \item Navigation in das Projektverzeichnis \texttt{/python/}.
    \item Erstellen der \texttt{conda}-Umgebung mithilfe der dort hinterlegten \texttt{environment.yml}:
          \begin{center}
              \texttt{conda env create -f environment.yml}
          \end{center}
    \item Aktivieren der \texttt{conda}-Umgebung:
          \begin{center}
              \texttt{conda activate spidercam\_simulator}
          \end{center}
    \item Installieren der \texttt{spidercam\_simulator}-Bibliothek:
          \begin{center}
              \texttt{pip install -e .}
          \end{center}
\end{enumerate}

% Das Modul kann dann wie folgt ausgeführt werden:
% \begin{center}
%     \texttt{python -m spidercam\_simulator}
% \end{center}

\subsubsection{Installation mit pip}
\label{sssec:installation_mit_pip}

Zur Installation mit \texttt{pip} wird eine verwendbare \texttt{python}-Instanz benötigt.

Empfohlen wird Python 3.10: \url{https://www.python.org/downloads/}.

Die Installation funktioniert dann wie folgt:
\begin{enumerate}
    \item Navigation in das Projektverzeichnis \texttt{/python/}.
    \item Installieren der benötigten Bibliotheken:
          \begin{center}
              \texttt{pip install -r requirements.txt}
          \end{center}
    \item Installieren der \texttt{spidercam\_simulator}-Bibliothek:
          \begin{center}
              \texttt{pip install -e .}
          \end{center}
\end{enumerate}

% Das Modul kann dann wie folgt ausgeführt werden:
% \begin{center}
%     \texttt{python -m spidercam\_simulator}
% \end{center}

\newpage
\subsection{Benutzung}
\label{ssec:benutzung}

Das Modul kann nach der Installation wie folgt ausgeführt werden:
\begin{center}
    \texttt{python -m spidercam\_simulator}
\end{center}

Dabei wird das Modul mit den Standardwerten ausgeführt, die in der Datei \texttt{config.ini} hinterlegt sind.
Es wird empfohlen, diese Datei nicht abzuändern.

Es werden standardmäßig alle Dateien aus dem Verzeichnis \texttt{python/input/} verarbeitet.
Die Ausgabe erfolgt in das Verzeichnis \texttt{python/output/}.
Erstellt werden:
\begin{itemize}
    \item Eine \texttt{.csv}-Datei entsprechend der Ausgabedatei 1, siehe \ref{ssec:ausgabe}.
          Sie endet auf \texttt{\_1.csv}.
    \item Eine \texttt{.csv}-Datei entsprechend der Ausgabedatei 2, siehe \ref{ssec:ausgabe}.
          Sie endet auf \texttt{\_2.csv}.
    \item Ein 3D-Plot der Bewegung der Spidercam.
          Die Datei endet auf \texttt{\_cam\_pos.png}.
    \item Ein 2D-Plot der Längen der Drahtseile.
          Die Datei endet auf \texttt{\_rope\_lengths.png}.
\end{itemize}

Folgende Argumente können an das Modul übergeben werden:
\begin{itemize}
    \item \texttt{--input, -i}: Pfad zu einer Eingabedatei oder einem Verzeichnis.
    \item \texttt{--output, -o}: Pfad zu einem Ausgabeverzeichnis.
    \item \texttt{--debug, -d}: Aktiviert den Debug-Modus.\footnote{Dabei werden zusätzliche Informationen geloggt.}
    \item \texttt{--no-plot, -np}: Deaktiviert die Erstellung der Plots.
\end{itemize}

Im Ordner \texttt{python/logs/} befindet sich eine Log-Datei, die Informationen über die Ausführung des Moduls enthält.
Sie wird bei jedem Programmstart überschrieben.

\subsubsection{Beispiele}
\label{sssec:beispiele}

\begin{itemize}
    \item Ausführen des Moduls mit den Standardwerten:
          \begin{center}
              \texttt{python -m spidercam\_simulator}
          \end{center}
    \item Ausführen des Moduls mit Debug-Modus:
          \begin{center}
              \texttt{python -m spidercam\_simulator -d}
          \end{center}
    \item Ausführen des Moduls mit Debug-Modus und Deaktivierung der Plots:
          \begin{center}
              \texttt{python -m spidercam\_simulator -d -np}
          \end{center}
    \item Ausführen des Moduls mit einer Eingabedatei:
          \begin{center}
              \texttt{python -m spidercam\_simulator -i input\_file.csv}
          \end{center}
    \item Ausführen des Moduls mit einem Eingabeverzeichnis:
          \begin{center}
              \texttt{python -m spidercam\_simulator -i input\_dir/}
          \end{center}
    \item Ausführen des Moduls mit einem Ausgabeverzeichnis:
          \begin{center}
              \texttt{python -m spidercam\_simulator -o output\_dir/}
          \end{center}
\end{itemize}

\subsubsection{Fehlermeldungen}
\label{sssec:fehlermeldungen}

Folgende Fehlermeldungen können auftreten:
\begin{itemize}
    \item \texttt{FileNotFoundError}: Eine angegebene Datei oder ein angegebenes Verzeichnis konnte nicht gefunden werden.
    \item \texttt{ValueError}: Eine Eingabedatei enthält ungültige Werte.
\end{itemize}

\subsubsection{Skripte}
\label{sssec:skripte}

Im Ordner \texttt{python/scripts/} befinden sich Skripte, die die Ausführung des Moduls erleichtern.
Sie können z. B. mit \texttt{sh scripts/\$name.sh} ausgeführt werden.

Folgende Skripte sind vorhanden und relevant:
\begin{itemize}
    \item \texttt{generate\_docs.sh}: Generiert die Entwicklerdokumentation.
    \item \texttt{generate\_uml.sh}: Generiert die UML-Klassendiagramme, die hier verwendet wurden.
    \item \texttt{run.sh}: Führt das Modul mit den Standardwerten aus. Es werden alle Dateien aus dem Verzeichnis \texttt{python/input/} verarbeitet.
    \item \texttt{run\_dev.sh}: Führt das Modul mit den Standardwerten aus. Es werden alle Dateien aus dem Verzeichnis \texttt{python/input/} verarbeitet.
          Zusätzlich wird der Debug-Modus aktiviert.
\end{itemize}

\section{Aufgabenanalyse}
\label{sec:aufgabenanalyse}

% Allgemeine Aufgabenanalyse
Im Rahmen dieses Softwareprojekts soll ein Programm erstellt werden, welches aus $\ldots$ möglichst effizient $\ldots$ erstellt.

Dabei soll das Programm insbesondere $\ldots$.

Die $\ldots$ werden aus einer Datei eingelesen und die $\ldots$ werden in einer anderen Datei ausgegeben.

Die besondere Schwierigkeit bei diesem Projekt liegt darin, dass $\ldots$.

Als generelles Vorgehen kommt $\ldots$ in Frage.

\subsection{Eingabe}
\label{ssec:eingabe}

% Beschreibung der Eingabe

\subsubsection{Format}
\label{sssec:format}

% Beschreibung des Formats der Eingabe
Für die Eingabe gilt, dass $\ldots$.
Ein Beispiel für eine Eingabedatei ist in Abbildung \ref{fig:example_input} zu sehen.

% https://www.matse-ausbildung.de/fileadmin/documents/pdf-files/Entwicklung_eines_Softwaresystems_13.12.07.pdf
% Path: figures/example_input.txt
\begin{figure}[H]
    \centering
    \lstinputlisting{figures/example_input.txt}
    \caption{Beispiel für eine Eingabedatei}
    \label{fig:example_input}
\end{figure}

% Beschreibung der Bedeutung der einzelnen Zeilen
Dabei gilt $\ldots$.

\subsubsection{Fehlerquellen und -behebung}
\label{sssec:fehlerquellen_und_-behebung}

% Beschreibung der Fehlerquellen
Die offensichtlichste Fehlerquellen sind:

\begin{itemize}
    \item $\ldots$
    \item $\ldots$
    \item $\ldots$
\end{itemize}

% Beschreibung der Fehlerbehebung
Um diese Fehlerquellen zu vermeiden, kann $\ldots$.

\subsection{Ausgabe}
\label{ssec:ausgabe}

% Beschreibung der Ausgabe

\subsubsection{Format}
\label{sssec:format}

% Beschreibung des Formats der Ausgabe
Für die Ausgabe gilt, dass $\ldots$.
Ein Beispiel für eine Ausgabedatei ist in Abbildung \ref{fig:example_output} zu sehen.

% https://www.matse-ausbildung.de/fileadmin/documents/pdf-files/Entwicklung_eines_Softwaresystems_13.12.07.pdf
% Path: figures/example_output.txt
\begin{figure}[H]
    \centering
    \lstinputlisting{figures/example_output.txt}
    \caption{Beispiel für eine Ausgabedatei}
    \label{fig:example_output}
\end{figure}

% Beschreibung der Bedeutung der einzelnen Zeilen
In der Ausgabedatei sollen also $\ldots$.

\subsubsection{Fehlerquellen und -behebung}
\label{sssec:fehlerquellen_und_-behebung}

% Beschreibung der Fehlerquellen
Die offensichtlichste Fehlerquellen sind:

\begin{itemize}
    \item $\ldots$
    \item $\ldots$
    \item $\ldots$
\end{itemize}

% Beschreibung der Fehlerbehebung
Um diese Fehlerquellen zu vermeiden, kann $\ldots$.
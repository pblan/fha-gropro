\section{Verfahrensbeschreibung}
\label{sec:verfahrensbeschreibung}

% Allgemeine Verfahrensbeschreibung
Für den zu simulierenden Sachverhalt lohnt es sich, jede Bewegung der Kamera in Phasen zu unterteilen, um unübersichtliche Strukturen zu vermeiden.
So bietet es sich an, eine \emph{Phase} (\texttt{Phase}) als entweder eine \emph{Beschleunigungs-} (\texttt{ACCELERATION}), \emph{Konstantgeschwindigkeits-} (\texttt{CONSTANT\_VELOCITY}) oder eine \emph{Bremsphase} (\texttt{DECELERATION}) zu definieren.

Eine \emph{Bewegung} (\texttt{Movement}) besteht dann aus entweder einer Beschleunigungs-, einer Konstantgeschwindigkeits- und einer Bremsphase oder nur aus einer Beschleunigungs- und einer Bremsphase.

Eine Instanz der Spidercam (\texttt{Spidercam}) enthält dann zusätzlich zu bekannten Konstanten eine Liste von Bewegungen (\texttt{movements}) und eine einelementige Warteschlange (\texttt{queue}).

\subsection{Datenstrukturen}
\label{ssec:datenstrukturen}

% Beschreibung der Datenstrukturen
Es werden folgende Datenstrukturen verwendet:

\begin{itemize}
    \item
\end{itemize}

\subsection{Algorithmen}
\label{ssec:algorithmen}

% Beschreibung der Algorithmen
Die genutzten Algorithmen unterteilen sich in zwei Kategorien:
\begin{itemize}
    \item Verarbeitung der Eingabedatei
    \item Erstellung der Ausgabedatei
\end{itemize}

\subsubsection{Verarbeitung der Eingabedatei}
\label{sssec:verarbeitung_der_eingabedatei}

% Beschreibung der Verarbeitung der Eingabe
Die Eingabe wird wie folgt verarbeitet:

\begin{enumerate}
    \item $\ldots$
    \item $\ldots$
    \item $\ldots$
\end{enumerate}

\subsubsection{Erstellung der Ausgabedatei}
\label{sssec:erstellung_der_ausgabedatei}

% Beschreibung der Erstellung der Ausgabe
Die Ausgabe wird wie folgt erstellt:

\begin{enumerate}
    \item $\ldots$
    \item $\ldots$
    \item $\ldots$
\end{enumerate}

Besonders nennenwert ist hier $\ldots$.

\section{Zusammenfassung und Ausblick}
\label{sec:zusammenfassung_und_ausblick}

% Zusammenfassung
\subsection{Zusammenfassung}
\label{ssec:zusammenfassung}

In dieser Arbeit wurde ein Programm zur Simulation einer Spidercam konzipiert und implementiert.
Die Simulation wurde in der Programmiersprache Python realisiert.
Entstanden ist ein Python-Modul, welches die Simulation einer Spidercam in einem beliebigen 3D-Raum ermöglicht.
Eingabedateien können sowohl einzeln, als auch automatisch nacheinander verarbeitet werden.

Alle relevanten Testfälle wurden erfolgreich durchlaufen.
Insbesondere wurde dies erkenntlich durch das Überprüfen der erstellten Logs in \texttt{python/logs/} und der erstellten Visualisierungen.

% Ausblick
\subsection{Ausblick}
\label{ssec:ausblick}

In Zukunft könnte man zur Optimierung beispielweise untersuchen, ob folgende Punkte Vorteile bringen:
\begin{itemize}
    \item Die Simulation in Echtzeit laufen lassen und Instruktionen live übertragen.
    \item Bei nötigen Bremsphasen die Zielkoordinaten anpassen, sodass kürzere Wege zurückgelegt werden müssen.
    \item Bei nötigen Brems- und Beschleunigungsphasen die Beschleunigung z. B. linear anpassen, damit die Bewegung nicht abrupt endet oder startet.
    \item Die Anwendung in eine GUI einbinden.
    \item Berechnungen parallelisieren.
    \item Eine noch performantere Programmiersprache verwenden.
\end{itemize}